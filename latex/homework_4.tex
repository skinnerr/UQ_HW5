\documentclass[11pt]{article}

%%
%% PACKAGES
%%

\usepackage[margin=0.9in, top=0.8in, bottom=1.0in]{geometry}
\usepackage[charter]{mathdesign} % Main font
\usepackage[scaled]{beramono} % Lovely monospace font
\usepackage[T1]{fontenc}
%\usepackage{amsmath, amssymb}
\usepackage{mathtools}
\usepackage{mathdots}
\usepackage{titlesec} % Custom section headings.
\usepackage{microtype}
\usepackage{xcolor}
\usepackage{xspace}
\usepackage{xfrac}
\usepackage{calc}
\usepackage{subcaption}

\usepackage{url}

\DeclareMathAlphabet{\mathc}{OMS}{cmsy}{m}{n}

% Graphics.
%\usepackage{graphicx}
\usepackage[update,prepend]{epstopdf} % To use eps files.

% Code listings.
\usepackage{listings} % Code listings.
\usepackage{matlab-prettifier} % MATLAB code listings

% Tweaks for captions and enumerations.
\usepackage[labelfont=bf]{caption} % Figure captions.
\usepackage{enumitem} % Fine tuning enumerations.
%\usepackage{floatrow} % Captions to the right of figures.

% Plotting and drawing
\usepackage{tikz} % This automatically loads graphicx!
\usetikzlibrary{calc} % For relative positions to defined coords
\usepackage{pgfplots} % Scientific plotting tools
\pgfplotsset{compat=1.7}

% Figure placement
%\usepackage{wrapfig}
%\captionsetup[wrapfigure]{margin=0.5cm}

% Packages to makes tables pretty.
\usepackage{array}
\usepackage{booktabs}
\setlength{\heavyrulewidth}{1.5pt}
\setlength{\abovetopsep}{4pt}

% Fancyhdr package stuff...
\usepackage{fancyhdr}
\setlength{\headheight}{0pt}
\setlength{\footskip}{50pt}
\renewcommand{\headrulewidth}{0pt}
\renewcommand{\footrulewidth}{0pt}

%%
%% SETTINGS
%%

% Path to look for graphics
\graphicspath{{../images/}}
%\epstopdfsetup{outdir=../images/}

% Caption spacing
\setlength{\abovecaptionskip}{0pt}

% List spacing
\setlist{noitemsep}

% Math operator font
\DeclareSymbolFont{sfoperators}{OT1}{cmss}{m}{n}
\DeclareSymbolFontAlphabet{\mathsf}{sfoperators}
\makeatletter
\def\operator@font{\mathgroup\symsfoperators}
\makeatother

%% No indent all paragraphs
%\setlength{\parindent}{0in}

% Figure references
\newcommand{\figref}[1]{Figure~\ref{#1}}

% Special format section headings
\titleformat{\section}%
	{\large\bf\scshape}% Text formatting
	{\arabic{section}}% Number
	{1em}% Space between number and text
	{}% Code before
	[]% Code after
\titleformat{\subsection}%
	{\normalsize\bf\scshape}% Text formatting
	{\arabic{section}.\arabic{subsection}}% Number
	{1em}% Space between number and text
	{}% Code before
	[]% Code after
%\titleformat{\subsubsection}%
%	{\color{blue}}% Text formatting
%	{\arabic{subsubsection} $\rightarrow$}% Number
%	{1em}% Space between number and text
%	{}% Code before
%	[]% Code after

\definecolor{mygray}{rgb}{0.4, 0.4, 0.4}
\lstset{
style=Matlab-editor,
mlscaleinline=false,
basicstyle=\ttfamily\lst@ifdisplaystyle\scriptsize\fi,
frame=single,
rulecolor=\color{mygray},
numbers=left,
numbersep=10pt,
numberstyle=\footnotesize \ttfamily \color{mygray},
xleftmargin=30pt,
xrightmargin=5pt,
framexleftmargin=4pt,
framextopmargin=2pt
}

% Allow white-space to be eaten within any lst environments between returns.
\lstset{breaklines,breakatwhitespace}

% Define the | chacacter as shorthand for inline listings.
%\lstMakeShortInline{`}

%%
%% COMMANDS
%%

% Various plot lines to include in-line.
\newcommand{\solidrule}[1][8mm]{\rule[0.5ex]{#1}{1.5pt}}
\newcommand{\dashrule}{\mbox{%
	\solidrule[2mm]\hspace{1mm}\solidrule[2mm]\hspace{1mm}\solidrule[2mm]}}
\newcommand{\dotdashrule}{\mbox{%
	\solidrule[0.5mm]\hspace{1mm}\solidrule[2mm]\hspace{1mm}\solidrule[0.5mm]\hspace{1mm}\solidrule[2mm]}}

% Automated file inclusion for code listings
\makeatletter
\def\includecode{\@ifnextchar[{\@with}{\@without}}
\def\@with[#1]#2{
}
\def\@without#1{
  \lstinputlisting[caption=\ttfamily\protect\detokenize{#1}, escapechar=, frame=single]{../matlab_code/#1}
}
\makeatother

% Degree symbol.
\newcommand{\degree}{\ensuremath{^\circ}}

% Superscript text: 1st, 2nd, 3rd, 4th
\newcommand{\suptext}[1]{\ensuremath{^\text{#1}}\xspace}
\newcommand{\st}{\suptext{st}}
\newcommand{\nd}{\suptext{nd}}
\newcommand{\rd}{\suptext{rd}}
\let\oldth\th % Reassign the current \th command
\renewcommand{\th}{\suptext{th}}

% Underline matrices
\newcommand{\ul}[1]{\smash{\underline{#1}}}
\newcommand{\uul}[1]{\smash{\underline{\underline{#1}}}}

% Partial derivatives
\newcommand{\pp}[2]{\ensuremath{\frac{\partial#1}{\partial#2}}}

% Math operators
\DeclareMathOperator\erf{erf}
\DeclareMathOperator\var{Var}

% Big O notation
\newcommand{\bigo}{\ensuremath{\mathcal{O}}}

% Text max and min
\newcommand{\tmax}{\ensuremath{\text{max}}}
\newcommand{\tmin}{\ensuremath{\text{min}}}

% Norm
\newcommand{\norm}[1]{\ensuremath{\left| #1 \right|}}

% Expectation
\newcommand{\xpect}[1]{\ensuremath{\left\langle #1 \right\rangle}}

% Bold vectors
% Option 1: Works on more than single tokens, but makes regular letters italic as well as bold.
%\renewcommand{\vec}[1]{\mathbold{#1}}
% Option 2: Only works if a single token is passed to the command, but makes regular letters bold only.
\newcommand{\mb}[1]{
	\ifcat\noexpand#1\relax
		\expandafter\mathbold
	\else
		\expandafter\mathbf
	\fi{{#1}}
}

% Underlines for tensor notation.
\newcommand{\tsr}[1]{\ensuremath{\underline{#1}}}
\newcommand{\tsrr}[1]{\ensuremath{\underline{\underline{#1}}}}


%%
%% DOCUMENT START
%%

\begin{document}


\newcommand{\widesim}[2][1.5]{
  \mathrel{\overset{#2}{\scalebox{#1}[1]{$\sim$}}}
}

\pagestyle{fancyplain}
\lhead{}
\chead{}
\rhead{}
\lfoot{\hrule UQ: Homework 4}
\cfoot{\hrule \thepage}
\rfoot{\hrule Ryan Skinner}

\noindent
{\Large Homework 4}
\hfill
{\large Ryan Skinner}
\\[0.5ex]
{\large ASEN 6519: Uncertainty Quantification}
\hfill
{\large Due 2016/04/19}\\
\hrule
\vspace{6pt}

%%%%%%%%%%%%%%%%%%%%%%%%%%%%%%%%%%%%%%%%%%%%%%%%%
%%%%%%%%%%%%%%%%%%%%%%%%%%%%%%%%%%%%%%%%%%%%%%%%%
\section*{Problem 1} %%%%%%%%%%%%%%%%%%%%%%%%%%%%
%%%%%%%%%%%%%%%%%%%%%%%%%%%%%%%%%%%%%%%%%%%%%%%%%
%%%%%%%%%%%%%%%%%%%%%%%%%%%%%%%%%%%%%%%%%%%%%%%%%

The thermal coefficient $K$ of a 1-D slab is defined on $\mathc{D} = (0,1)$ and is characterized by a shifted log-normal random process
\begin{equation}
K(x,\omega) = 2 + \exp(G(x,\omega)), \quad x \in \mathc{D}
\end{equation}
where $G(x,\omega)$ is a Gaussian random process also defined on $\mathc{D}$. The mean and covariance functions of $G(x,\omega)$ are
\begin{equation}
\xpect{G(x,\cdot)} = 1.0, \quad x \in \mathc{D}
\end{equation}
and
\begin{equation}
C_{GG}(x_1,x_2) = \sigma^2 \exp \left( \frac{- | x_1-x_2 | }{\ell} \right), \quad (x1,x2) \in \mathc{D} \times \mathc{D}
\end{equation}
respectively. We would like to compute the statistics of the temperature field $u(x)$ by solving the governing steady-state stochastic heat equation
\begin{equation}
\begin{aligned}
-\pp{}{x} \left( K(x,\omega) \pp{u(x,\omega)}{x} \right) &= 1.0, \qquad x \in \mathc{D}, \\
u(0,\omega) &= 0, \\
u(1,\omega) &= 0,
\end{aligned}
\label{eq:pde}
\end{equation}
using the stochastic Galerkin approach.

For this problem, we let $\sigma = 2$ and $\ell = 2.0$, and use the codes from Homework \#1 to generate a $d=2$ Karhunen-L\"oeve expansion of $G(x,\omega)$ via the analytical solution. The approximate log-normal is then computed through its polynomial chaos expansion (PCE)
\begin{equation}
K_{p_k}(x,\mb{y}) = \sum_{i=0}^{P_K} K_i(x) \Psi_i(\mb{y})
\end{equation}
where $K_0$ and $K_i$ ($i>0$) are defined as
\begin{equation}
K_0 = 2 + \exp \left( 1 + \frac{1}{2} \sum_{j=1}^d \lambda_j \phi_j^2(x) \right)
\end{equation}
and
\begin{equation}
K_i(x) = \frac{K_0(x) - 2}{\sqrt{\sum_{j=1}^d (i_j!)}} \sum_{j=1}^d \left( \sqrt{\lambda_j} \phi_j(x) \right)^{i_j}
\end{equation}
where $\{ \lambda_i, \phi_i(x) \}_{i=1}^d$ are eigen-pairs of the covariance kernel $C_{GG}$, and $i_j$ denotes the polynomial order of $\Psi_i(\mb{y})$ along the direction $j \in \{1, \dots, d\} = \{1,2\}$. We set $p_k = 14$ as the total order of the PC expansion, and write a finite difference code to solve the stochastic PDE in \eqref{eq:pde}. This employs the stochastic Galerkin discretization of total order $p$
\begin{equation}
u_p(x,\mb{y}) = \sum_{j=0}^{P} u_j(x) \Psi_j(\mb{y})
\label{eq:galerkin_discretization}
\end{equation}

\subsection*{Solution}

In the stochastic Galerkin discretization, we first discretize our differential operator in space, extracting the stochastic component from \eqref{eq:pde} through the PCE and arriving at
\begin{align}
\sum_{k=0}^{P_K}
\;\;
\frac{-\partial}{\partial x} \left(K_k(x) \pp{\cdot}{x} \right)
\;
u(x,\mb{y})
\;
\Psi_k(\mb{y})
&=
f
\\
\sum_{k=0}^{P_K}
\;\;
\uul{K}_k
\Psi_k(\mb{y})
\;
\ul{u}(x,\mb{y})
&=
\ul{f}
\end{align}
Combining the sum over $k$ into one composite matrix yields the pseudo-spectral system
\begin{equation}
\uul{K}(\mb{y}) \; \ul{u}(x,\mb{y}) = \ul{f}
\end{equation}

We then use another PC expansion on the solution $\ul{u}(x,\mb{y})$, approximating it as
\begin{equation}
\ul{\hat{u}}(\mb{y}) = \sum_{j=0}^P \ul{u}_j \Phi_j(\mb{y})
\end{equation}
Then, noting that the optimal solution's residual must be orthogonal to the PCE basis functions, we arrive at the system of equations
\begin{align}
\xpect{\uul{K}(\mb{y}) \; \ul{\hat{u}}(\mb{y}) - \ul{f}, \Psi_i(\mb{y})} &= 0
\\
\sum_{j=0}^P
\xpect{\uul{K}(\mb{y}) \; \Psi_i(\mb{y}) \; \Psi_j(\mb{y}) } \ul{u}_j(\mb{y}) &= \xpect{\ul{f}, \Psi_i(\mb{y})}
\end{align}
which can be written in matrix form as
\begin{equation}
\begin{bmatrix}
\uul{K}_{0,0} & \cdots & \uul{K}_{0,P} \\
\vdots & \ddots & \vdots \\
\uul{K}_{P,0} & \cdots & \uul{K}_{P,P}
\end{bmatrix}
\begin{bmatrix}
\ul{u}_0 \\ \vdots \\ \ul{u}_P
\end{bmatrix}
=
\begin{bmatrix}
\ul{f}_0 \\ \vdots \\ \ul{f}_P
\end{bmatrix}
\end{equation}
where
\begin{align}
\uul{K}_{i,j} &\equiv \sum_{k=0}^{P_K} \; \uul{K}_k \overbrace{\xpect{\Psi_i \Psi_j \Psi_k}}^{c_{ijk}}
\\
\ul{f}_i &=
\begin{cases}
\ul{f} & i=0 \\
\ul{0} & \text{else}
\end{cases}
\end{align}

What makes the polynomial chaos approach so attractive is that, though it is intrusive and requires knowledge of the PDE being solved, components of an existing numerical solver can be leveraged. In this example, the $\uul{K}_k$ matrices and the $\ul{f}$ vector are the LHS and RHS of the matrix system constructed by a standard central difference heat equation code.

Additionally, because $c_{ijk} = 0$ when $k > 2P$, it is the case that if $p_k \ge 2p$, the Galerkin projection of the PDE with the thermal coefficient expansion truncated at $P_K$ terms ($K_{p_k}$) is equivalent to the same expansion including infinite terms ($K_{p_k=\infty}$). Thus, no benefit is gained from increasing $p_k$ beyond $2p$. Because we take $p \in \{0, 1, \dots, 7\}$ in this homework with $p_k=14$, no computational effort is wasted.

The convergence of our solution is shown in Figure \ref{fig:1ab}, wherein for a fixed total order $p_k$ of the thermal coefficient $K_{p_k}$ PC expansion, the the total order $p$ of the Galerkin projection solution goes from 0 to 7. Both mean and variance, which are computed from the PCE as
\begin{equation}
\begin{aligned}
\xpect{u(x)}   &\approx u_0(x) \\
\var(x) &= \xpect{u(x)^2} - \xpect{u(x)}^2, \quad
\xpect{u(x)^2} \approx \sum_{i=0}^P u_i(x)^2
\end{aligned}
\end{equation}
are seen to approach a limiting function of $x$ as $p$ is increased.

To calculate a ``true'' mean and variance, we use the tensor-product stochastic collocation approach with $n_q=40$ quadrature points in each dimension and a Gauss-Hermite quadrature rule. This procedure is relatively simple, in that we sample the value of $\Psi_j(\mb{y})$ at the value of $\mb{y}$ indicated by the abscissas, and then compute the first and second moments of $u$ in the manner of \eqref{eq:galerkin_discretization}. The mean and variance are related to these moments as above, and are plotted in Figure \ref{fig:1c_true}. This enables us to calculate and plot the relative error in the mean and variance as a function of $x$, and for $x=0.5$, as $p$ is increased from 0 to 7. These results are shown in Figures \ref{fig:1c_relerr_full} and \ref{fig:1c_relerr_pt5}.

We would expect a much smoother mean and variance, which is reason enough to believe that the ``true'' mean and variance are not the exact solutions. However, they are sufficient for our purposes of approximating a ``true'' solutions. The cool thing about the PCE approach is that we've only solved a single linear system, rather than solving $40^2 = 1600$ different realizations of the PDE. Further reason to suspect the truth of this solution, or the accuracy of our method at capturing variance, is seen in Figure \ref{fig:1c_relerr_pt5}, which shows that as $p$ increases past 3, little improvement in the approximation of $\var(u(x=0.5))$ is attained.

\begin{figure}[p]
\centering
\includegraphics[width=0.8\textwidth]{prob_1ab.eps}
\caption{Convergence of temperature field mean and variance as the order $p$ of the Galerkin projection is increased.}
\label{fig:1ab}
\end{figure}

\begin{figure}[p]
\centering
\includegraphics[width=0.8\textwidth]{prob_1c_true.eps}
\caption{``True'' mean and variance of the temperature field, as computed from Gauss-Hermite tensor-product stochastic collocation.}
\label{fig:1c_true}
\end{figure}

\begin{figure}[p]
\centering
\includegraphics[width=0.8\textwidth]{prob_1c_relerr_full.eps}
\caption{Relative error as a function of position computed from data in Figures \ref{fig:1ab} and \ref{fig:1c_true}.}
\label{fig:1c_relerr_full}
\end{figure}

\begin{figure}[p]
\centering
\includegraphics[width=0.8\textwidth]{prob_1c_relerr_pt5.eps}
\\[1em]
\caption{Relative error at $x=0.5$ computed from data in Figures \ref{fig:1ab} and \ref{fig:1c_true}.}
\label{fig:1c_relerr_pt5}
\end{figure}


%%
%% DOCUMENT END
%%
\end{document}
