\documentclass[11pt]{article}

%%
%% PACKAGES
%%

\usepackage[margin=0.9in, top=0.8in, bottom=1.0in]{geometry}
\usepackage[charter]{mathdesign} % Main font
\usepackage[scaled]{beramono} % Lovely monospace font
\usepackage[T1]{fontenc}
%\usepackage{amsmath, amssymb}
\usepackage{mathtools}
\usepackage{mathdots}
\usepackage{titlesec} % Custom section headings.
\usepackage{microtype}
\usepackage{xcolor}
\usepackage{xspace}
\usepackage{xfrac}
\usepackage{calc}
\usepackage{subcaption}

\usepackage{url}

\DeclareMathAlphabet{\mathc}{OMS}{cmsy}{m}{n}

% Graphics.
%\usepackage{graphicx}
\usepackage[update,prepend]{epstopdf} % To use eps files.

% Code listings.
\usepackage{listings} % Code listings.
\usepackage{matlab-prettifier} % MATLAB code listings

% Tweaks for captions and enumerations.
\usepackage[labelfont=bf]{caption} % Figure captions.
\usepackage{enumitem} % Fine tuning enumerations.
%\usepackage{floatrow} % Captions to the right of figures.

% Plotting and drawing
\usepackage{tikz} % This automatically loads graphicx!
\usetikzlibrary{calc} % For relative positions to defined coords
\usepackage{pgfplots} % Scientific plotting tools
\pgfplotsset{compat=1.7}

% Figure placement
%\usepackage{wrapfig}
%\captionsetup[wrapfigure]{margin=0.5cm}

% Packages to makes tables pretty.
\usepackage{array}
\usepackage{booktabs}
\setlength{\heavyrulewidth}{1.5pt}
\setlength{\abovetopsep}{4pt}

% Fancyhdr package stuff...
\usepackage{fancyhdr}
\setlength{\headheight}{0pt}
\setlength{\footskip}{50pt}
\renewcommand{\headrulewidth}{0pt}
\renewcommand{\footrulewidth}{0pt}

%%
%% SETTINGS
%%

% Path to look for graphics
\graphicspath{{../images/}}
%\epstopdfsetup{outdir=../images/}

% Caption spacing
\setlength{\abovecaptionskip}{0pt}

% List spacing
\setlist{noitemsep}

% Math operator font
\DeclareSymbolFont{sfoperators}{OT1}{cmss}{m}{n}
\DeclareSymbolFontAlphabet{\mathsf}{sfoperators}
\makeatletter
\def\operator@font{\mathgroup\symsfoperators}
\makeatother

%% No indent all paragraphs
%\setlength{\parindent}{0in}

% Figure references
\newcommand{\figref}[1]{Figure~\ref{#1}}

% Special format section headings
\titleformat{\section}%
	{\large\bf\scshape}% Text formatting
	{\arabic{section}}% Number
	{1em}% Space between number and text
	{}% Code before
	[]% Code after
\titleformat{\subsection}%
	{\normalsize\bf\scshape}% Text formatting
	{\arabic{section}.\arabic{subsection}}% Number
	{1em}% Space between number and text
	{}% Code before
	[]% Code after
%\titleformat{\subsubsection}%
%	{\color{blue}}% Text formatting
%	{\arabic{subsubsection} $\rightarrow$}% Number
%	{1em}% Space between number and text
%	{}% Code before
%	[]% Code after

\definecolor{mygray}{rgb}{0.4, 0.4, 0.4}
\lstset{
style=Matlab-editor,
mlscaleinline=false,
basicstyle=\ttfamily\lst@ifdisplaystyle\scriptsize\fi,
frame=single,
rulecolor=\color{mygray},
numbers=left,
numbersep=10pt,
numberstyle=\footnotesize \ttfamily \color{mygray},
xleftmargin=30pt,
xrightmargin=5pt,
framexleftmargin=4pt,
framextopmargin=2pt
}

% Allow white-space to be eaten within any lst environments between returns.
\lstset{breaklines,breakatwhitespace}

% Define the | chacacter as shorthand for inline listings.
%\lstMakeShortInline{`}

%%
%% COMMANDS
%%

% Various plot lines to include in-line.
\newcommand{\solidrule}[1][8mm]{\rule[0.5ex]{#1}{1.5pt}}
\newcommand{\dashrule}{\mbox{%
	\solidrule[2mm]\hspace{1mm}\solidrule[2mm]\hspace{1mm}\solidrule[2mm]}}
\newcommand{\dotdashrule}{\mbox{%
	\solidrule[0.5mm]\hspace{1mm}\solidrule[2mm]\hspace{1mm}\solidrule[0.5mm]\hspace{1mm}\solidrule[2mm]}}

% Automated file inclusion for code listings
\makeatletter
\def\includecode{\@ifnextchar[{\@with}{\@without}}
\def\@with[#1]#2{
}
\def\@without#1{
  \lstinputlisting[caption=\ttfamily\protect\detokenize{#1}, escapechar=, frame=single]{../matlab_code/#1}
}
\makeatother

% Degree symbol.
\newcommand{\degree}{\ensuremath{^\circ}}

% Superscript text: 1st, 2nd, 3rd, 4th
\newcommand{\suptext}[1]{\ensuremath{^\text{#1}}\xspace}
\newcommand{\st}{\suptext{st}}
\newcommand{\nd}{\suptext{nd}}
\newcommand{\rd}{\suptext{rd}}
\let\oldth\th % Reassign the current \th command
\renewcommand{\th}{\suptext{th}}

% Underline matrices
\newcommand{\ul}[1]{\smash{\underline{#1}}}
\newcommand{\uul}[1]{\smash{\underline{\underline{#1}}}}

% Partial derivatives
\newcommand{\pp}[2]{\ensuremath{\frac{\partial#1}{\partial#2}}}

% Math operators
\DeclareMathOperator\erf{erf}
\DeclareMathOperator\var{Var}

% Big O notation
\newcommand{\bigo}{\ensuremath{\mathcal{O}}}

% Text max and min
\newcommand{\tmax}{\ensuremath{\text{max}}}
\newcommand{\tmin}{\ensuremath{\text{min}}}

% Norm
\newcommand{\norm}[1]{\ensuremath{\left| #1 \right|}}

% Expectation
\newcommand{\xpect}[1]{\ensuremath{\left\langle #1 \right\rangle}}

% Bold vectors
% Option 1: Works on more than single tokens, but makes regular letters italic as well as bold.
%\renewcommand{\vec}[1]{\mathbold{#1}}
% Option 2: Only works if a single token is passed to the command, but makes regular letters bold only.
\newcommand{\mb}[1]{
	\ifcat\noexpand#1\relax
		\expandafter\mathbold
	\else
		\expandafter\mathbf
	\fi{{#1}}
}

% Underlines for tensor notation.
\newcommand{\tsr}[1]{\ensuremath{\underline{#1}}}
\newcommand{\tsrr}[1]{\ensuremath{\underline{\underline{#1}}}}


%%
%% DOCUMENT START
%%

\begin{document}


\newcommand{\widesim}[2][1.5]{
  \mathrel{\overset{#2}{\scalebox{#1}[1]{$\sim$}}}
}

\pagestyle{fancyplain}
\lhead{}
\chead{}
\rhead{}
\lfoot{\hrule UQ: Homework 5}
\cfoot{\hrule \thepage}
\rfoot{\hrule Ryan Skinner}

\noindent
{\Large Homework 5}
\hfill
{\large Ryan Skinner}
\\[0.5ex]
{\large ASEN 6519: Uncertainty Quantification}
\hfill
{\large Due 2016/04/28}\\
\hrule
\vspace{6pt}

%%%%%%%%%%%%%%%%%%%%%%%%%%%%%%%%%%%%%%%%%%%%%%%%%
%%%%%%%%%%%%%%%%%%%%%%%%%%%%%%%%%%%%%%%%%%%%%%%%%
\section*{Problem 1} %%%%%%%%%%%%%%%%%%%%%%%%%%%%
%%%%%%%%%%%%%%%%%%%%%%%%%%%%%%%%%%%%%%%%%%%%%%%%%
%%%%%%%%%%%%%%%%%%%%%%%%%%%%%%%%%%%%%%%%%%%%%%%%%

The thermal coefficient $K$ of a 1-D slab is defined on $\mathc{D} = (0,1)$ and is characterized by a shifted log-normal random process
\begin{equation}
K(x,\omega) = 2 + \exp(G(x,\omega)), \quad x \in \mathc{D}
\end{equation}
where $G(x,\omega)$ is a Gaussian random process also defined on $\mathc{D}$. The mean and covariance functions of $G(x,\omega)$ are
\begin{equation}
\xpect{G(x,\cdot)} = 1.0, \quad x \in \mathc{D}
\end{equation}
and
\begin{equation}
C_{GG}(x_1,x_2) = \sigma^2 \exp \left( \frac{- | x_1-x_2 | }{\ell} \right), \quad (x1,x2) \in \mathc{D} \times \mathc{D}
\end{equation}
respectively. We would like to compute the statistics of the temperature field $u(x)$ by solving the governing steady-state stochastic heat equation
\begin{equation}
\begin{aligned}
-\pp{}{x} \left( K(x,\omega) \pp{u(x,\omega)}{x} \right) &= 1.0, \qquad x \in \mathc{D}, \\
u(0,\omega) &= 0, \\
u(1,\omega) &= 0,
\end{aligned}
\label{eq:pde}
\end{equation}
using the \textbf{least-squares regression} approach.

For this problem, we let $\sigma = 2$ and $\ell = 2.0$, and use the codes from Homework \#1 to generate a $d=2$ Karhunen-Lo\`eve expansion of $G(x,\omega)$ via the analytical solution. The approximate log-normal is then computed through its polynomial chaos expansion (PCE)
\begin{equation}
K_{p_k}(x,\mb{y}) = \sum_{i=0}^{P_K} K_i(x) \Psi_i(\mb{y})
\label{eq:k_pce}
\end{equation}
where $K_0$ and $K_i$ ($i>0$) are defined as
\begin{equation}
K_0 = 2 + \exp \left( 1 + \frac{1}{2} \sum_{j=1}^d \lambda_j \phi_j^2(x) \right)
\end{equation}
and
\begin{equation}
K_i(x) = \frac{K_0(x) - 2}{\sqrt{\sum_{j=1}^d (i_j!)}} \sum_{j=1}^d \left( \sqrt{\lambda_j} \phi_j(x) \right)^{i_j}
\end{equation}
where $\{ \lambda_i, \phi_i(x) \}_{i=1}^d$ are eigen-pairs of the covariance kernel $C_{GG}$, and $i_j$ denotes the polynomial order of $\Psi_i(\mb{y})$ along the direction $j \in \{1, \dots, d\} = \{1,2\}$. We set $p_k = 14$ as the total order of the PC expansion, and write a finite difference code to solve the PDE in \eqref{eq:pde} for a given $K_{p_k}$, which we are able to compute for arbitrary $\mb{y}$. Note that $\mb{y}$ is a Gaussian random vector. We are only interested in the values of our solution at $x=0.5$, so we employ the stochastic Galerkin discretization of total order $p$ (total number of terms $P$):
\begin{equation}
u_p(\mb{y}) = \sum_{j=0}^{P} \underbrace{u_j(x=0.5)}_{c_j} \Psi_j(\mb{y})
\label{eq:galerkin_discretization}
\end{equation}

\subsection*{Solution}

In the least-squares linear regression approach, we sample random vectors $\mb{y}_1, \mb{y}_2, \dots, \mb{y}_N$. For each $\mb{y}_i$, we compute the solution $u(x=0.5,\mb{y}_i)$ and the basis functions $\Psi_j(\mb{y}_i)$. These are used to form our LHS measurement matrix and RHS solution sample vector in a linear system that we seek to solve for the coefficient vector $\ul{c}$:

\begin{equation}
\underbrace{
\begin{bmatrix}
\psi_1(\mb{y}_1) & \cdots & \psi_P(\mb{y}_1) \\
\vdots & \ddots & \vdots \\
\psi_1(\mb{y}_N) & \cdots & \psi_P(\mb{y}_N)
\end{bmatrix}
}_{\mb{\Psi}}
\underbrace{
\begin{bmatrix}
\hat{c}_1 \\ \vdots \\ \hat{c}_P
\end{bmatrix}
}_{\mb{\hat{c}}}
=
\underbrace{
\begin{bmatrix}
u(\mb{y}_1) \\ \vdots \\ u(\mb{y}_N)
\end{bmatrix}
}_{\mb{u}}
\end{equation}

The least squares solution is obtained by solving the matrix system
\begin{equation}
( \mb{\Psi}^T \mb{\Psi} ) \mb{\hat{c}} = \mb{\Psi}^T \mb{u}
\end{equation}
which is over-determined if $N > P$ and under-determined and susceptible to instability if $N < P$.

To construct the matrix, we loop over $N$ samples of our Gaussian random variable $\mb{y}$. For each sample $\mb{y}_i$, we compute the values of $\Psi_1, \dots, \Psi_P$ and place them in the $\mb{\Psi}$-matrix. The heat equation solution is computed with the same realization of $\mb{y}_i$. However, note that no solution PCE basis functions are used in the computation of $u(\mb{y}_i)$, only the basis functions used to represent $K_{p_k}$ per \eqref{eq:k_pce}.

The results for total solution polynomial order of $p=3$ ($P=10$) and $N$ ranging from $0.5 P$ to more than $4 P$ are shown in Figure \ref{fig:1b}. We see the mean and variance approach the values we obtained using the stochastic Galerkin approach in Homework \#4. It appears that a converged variance is reached with approximately $N=4P=40$ samples, which is in keeping with the recommended value of $(4\text{--}6)P$ that was discussed in class.

Looking at the relative error in the variance from \ref{fig:1c}, we see that as the polynomial order of the stochastic Galerkin PC expansion increases from $p=1$ to $p=4$, the solution improves in accuracy if for each employ $N=4P$ samples. However, this is just one realization of the plot; even when averaging 1000 $N$-sample computations for variance, our relative error was not always monotonically decreasing.

\begin{figure}[p]
\centering
\includegraphics[width=0.9\textwidth]{prob1_b.eps}
\caption{For total solution polynomial order $p=3$, convergence in the mean and variance of $u(x=0.5,\mb{y})$ as a function of samples $N$.}
\label{fig:1b}
\end{figure}

\begin{figure}[p]
\centering
\includegraphics[width=0.8\textwidth]{prob1_c.eps}
\caption{Relative error of $\var(u(x=0.5,\cdot))$ as the solution's total polynomial order $p$ is increased. $N=4P$ realizations are computed for each $p$ to achieve an over-sampled system for linear regression.}
\label{fig:1c}
\end{figure}


%%
%% DOCUMENT END
%%
\end{document}
